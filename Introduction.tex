%!TeX root=Main Document/Dissertation.tex

\section{Introduction}

\subsection{General Introduction and Background Information} % May not be entirely necessary to title this this way and can probably be skipped (depends how you write this though!)

Flocking is a behaviour in which all social organisms engage; it is the common movement of organisms guided by both social and environmental pressures. 
These flocks of organisms can be found interacting with other flocks in nature, and the way they interact is as varied as it is interesting.

Producing flocking behavior that recreates those found in nature is an endeavor already undertaken and in constant update. The field, taking off in 1987 with Craig Reynolds’ influential paper \citep{Reynolds:1987:FHS:37402.37406}, shows how realistic flocking behaviour can already be achieved by applying 3 simple rules to each boid (‘boid’ as coined in the prior paper, this is essentially an agent) in regard to its neighbours: Cohesion, movement toward the average position; Alignment, movement toward the average direction; and Separation, movement to avoid collision. 

Since then, the original flocking algorithm has been extended in various ways, with communication techniques, mathematical models for how leadership arises in the flock, as well as models for how consensus is made in a flock. These expansions allow for more complex behaviours and reactions to the enviornment and surroundings.

Learning behaviours have also been added. The behaviour and strategies flocks produce are patterns. This means if a flock can learn and understand those patterns it has a significant advantage over that other flock – an interesting example would be a group of honey hunters and smoke; bees flee the nest if they think there is a fire, the first warning sign to this is smoke, as a group (or flock) they take advantage of this by releasing smoke into the hive. The bees evacuate the nest; they get honey. 

This dissemination of new knowledge, either through behaviour or communication is interesting because it can increase the complexity of the reactions the flock has to a given situation. This added complexity may lead to new behaviours that may not have been easily predicted or thought of as something a flock could produce. 


\subsection{Aim} 
To investigate the impact of a genetic algorithm on the dynamic interaction of flocks with each other to see if this has a beneficial effect in comparison to regular flocking algorithms.

\subsection{Objectives} % These objectives need to be updated as they are not reflective of the project in its current form
\begin{itemize}
\item % This item especially needs updating!
To research and evaluate AI techniques, studying their relevance and potential for further development in applying them to flocking algorithms, with particular focus on artificial life techniques. 
\item
To produce an application that models flocking behaviour, and allows the observation and comparison of AI flocking strategies to regular flocking algorithms. This will be developed using the prior identified AI techniques most likely to produce viable flocking behaviours. 
\item
To analyse the effectiveness of strategies that the AI come up with in their interactions with other flocks, comparing and contrasting that to the behavior of standard flocking algorithms.
\end{itemize}

\subsection{Research Question}
What impact can a genetic algorithm have on the dynamic interaction of flocks with each other in comparison to that of regular flocking techniques?


